%========Top Matter=========== (fold)
\documentclass[final,expand]{problemset}
% For plot: pfdplots and settings
\usepackage{pgfplots}
\pgfplotsset{width=7cm,
  compat=newest,
  label style={font=\small},
  legend style={font=\small}
}
%========top_matter=========== (end)

\begin{document}
\heading[Sebastián Caballero]{Sebastián Caballero}{2023}{Problem set}{Self Study}

\section*{Vector Spaces I}

\problem
Let $V$ and $W$ be vector spaces over a field $K$. Let $\mathcal{B} = \{v_1, v_2, \dots,
	v_n\}$ be a basis for $V$ and let $\{w_1, w_2, \dots, w_n\}$ be any
vectors in
$W$. There is a unique linear map \begin{align*}
	\begin{matrix}
		\phi: & V & \to & W \\
	\end{matrix}
\end{align*}
Such that $\phi(v_i) = w_i$ for all $1 \le i \le n$

\solution{
	Since $\mathcal{B}$ is a basis for $V$, for any element $v \in V$ there are $a_1, a_2, \dots, a_n \in K$ such that:
	\begin{align*}
		v &= a_1v_1 + a_2v_2 + \dots + a_nv_n
	\end{align*}
	so if we define $\phi$ such that $\phi(v_i) = w_i$ then for any vector $v$ we would have:
	\begin{align*}
		\phi(v) &= a_1\phi(v_1) + a_2\phi(v_2) + \dots + a_n\phi(a_n)\\
		&= a_1w_1 + a_2w_2 + \dots + a_nw_n
	\end{align*}
}

\problem
Suppose that $V$ is a finite dimensional vector space. Let $\mathcal{B} = \{v_1, v_2, \dots, v_n\}$ be a basis for $V$ then:
\begin{itemize}
	\item Any set of $w_1, w_2, \dots, w_n, w_{n+1}$ vectors is linearly dependent
	\item Any set of $w_1, w_2, \dots, w_{n-1}$ vectors can't generate $V$
\end{itemize}

\solution{
	For this, we are going to use the facts needed for a basis.
	\begin{itemize}
		\item Let $w_1, w_2, \dots, w_n, w_{n+1}$ be vectors in $V$, we can write them in the next way:
		\begin{align*}
			w_1 &= a_{1, 1}v_1 + a_{1, 2}v_2 + \dots + a_{1, n} v_n \\
			w_2 &= a_{2, 1}v_1 + a_{2, 2}v_2 + \dots + a_{2, n} v_n  \\
			\dots & \dots \\
			w_n &= a_{n, 1}v_1 + a_{n, 2}v_2 + \dots + a_{n, n} v_n \\
			w_{n+1} &= a_{n+1, 1}v_1 + a_{n+1, 2}v_2 + \dots + a_{n+1, n} v_n \\
		\end{align*}
		If there is a $w_i$ such that $w_i = 0$ we are done. Suppose then that this is not true, so for each $1 \le i \le n + 1$ exists $j$ such that $a_{i, j} \neq 0$. But since there are $w_{n+1}$ there must be $i_1, i_2$ such that for the same $j$, we have that $a_{i_1, j} \neq 0 \neq a_{i_2, j}$. So, we can express the vector $v_j$ as:
		\begin{align*}
			v_j &= \frac{w_{i_1}}{a_{i_1, j}} - \frac{a_{i_1, 1}v_1 + a_{i_1, 2}v_2 + \dots + a_{i_1, n}v_n}{a_{i_1, j}}\\
			v_j &= \frac{w_{i_2}}{a_{i_2, j}} -\frac{a_{i_2, 1}v_1 + a_{i_2, 2}v_2 + \dots + a_{i_2, n}v_n}{a_{i_2, j}}
		\end{align*}
		And so the set is not linearly independent.

		\item Let $w_1, w_2, \dots, w_{n-1}$ be vectors of $V$. Suppose that indeed we can generate $V$ with them, so in particular, we can write:
		\begin{align*}
			v_1 &= a_{1, 1}w_1 + a_{1, 2}w_2 + \dots + a_{1, {n-1}} w_{n-1} \\
			v_2 &= a_{2, 1}w_1 + a_{2, 2}w_2 + \dots + a_{2, {n-1}} w_{n-1}  \\
			\dots & \dots \\
			v_n &= a_{n, 1}w_1 + a_{n, 2}w_2 + \dots + a_{n, {n-1}} w_{n-1} \\
		\end{align*}
		And since none of them is zero, we can be fure that for each $1 \le i \le n$ exists $j$ such that $a_{i, j} \neq 0$. But since there are $n$ vectors in $\mathcal{B}$ and just $n-1$ vectors $w_i$, there must be $i_1, i_2$ such that for the same $j$, we have that $a_{i_1, j} \neq 0 \neq a_{i_2, j}$. So, we can express the vector $v_j$ as:
		\begin{align*}
			w_j &= \frac{v_{i_1}}{a_{i_1, j}} - \frac{a_{i_1, 1}w_1 + a_{i_1, 2}w_2 + \dots + a_{i_1, n}w_n}{a_{i_1, j}}\\
			w_j &= \frac{v_{i_2}}{a_{i_2, j}} -\frac{a_{i_2, 1}w_1 + a_{i_2, 2}w_2 + \dots + a_{i_2, n}w_n}{a_{i_2, j}}
		\end{align*}
		But then this let us generate two different linear combinations within $\mathcal{B}$ that give us the same result, contradicting the linear independency of $\mathcal{B}$.
	\end{itemize}
}

\problem Let $V$ be a finite vector space. If $A = \{v_1, v_2, \dots, v_n\}$ generates $V$ then some subset of $A$ is a basis for $V$.

\solution{
    For that, let declare the next set:
    \begin{align*}
        S &= \{W \in \mathcal{P}(A) | W \text{ is linearly independent}\}
    \end{align*}
    We can assure that at least there is a maximal element $\{v_1, v_2, \dots, v_m\}$ in $S$ since we can assure the existence of $\{v_1\}$ and at most it can be $A$. Suppose then that it is not $A$, so $m < n$, and we can assure that any set $\{v_1, \dots, v_m, v_i\}$ is linearly dependent, with $m < i \le n$. Therefore we have:
	\begin{align*}
		a_1v_1 + \dots + a_nv_n + a_iv_i &= 0
	\end{align*}
	has more than the trivial solution, so we can suppose that 
}

\problem
Let $A = \{v_1, v_2, \dots, v_n\}$ be a subset of a vector space $V$. Prove that $A$ is linearly independent if and only if the equation $a_1v_1 + a_2v_2 + \dots + a_nv_n = 0$ has the trivial solution.

\solution{
	We prove a double implication:
	\begin{itemize}
		\item[$\Rightarrow)$] If $A$ is linearly independent then by definition the equation $a_1v_1 + a_2v_2 + \dots + a_nv_n = 0$ has only one solution, the trivial one.
		\item[$\Leftarrow)$] Suppose that $A$ is not linearly independent, so that there are two combinations of scalars $a_1, a_2, \dots, a_n$ and $b_1, b_2, \dots, b_n$ such that for a $v$ in $V$:
		\begin{align*}
			a_1v_1 + a_2v_2 + \dots + a_nv_n &= v\\
			b_1v_1 + b_2v_2 + \dots + b_nv_n &= v
		\end{align*}
		And if we use the transitivity we have:
		\begin{align*}
			a_1v_1 + a_2v_2 + \dots + a_nv_n &= b_1v_1 + b_2v_2 + \dots + b_nv_n\\
			(a_1 - b_1)v_1 + (a_2 - b_2)v_2 + \dots + (a_n - b_n)v_n &= 0
		\end{align*} 
		But note that $a_1 \neq b_1$, $a_2 \neq b_2$ and so on, so $a_1 - b_1 \neq 0$, $a_2 - b_2 \neq 0$ and so on, so the equation has another solution apart to the trivial one.
	\end{itemize}
}

\problem
Prove the Rank theorem

\solution{
	Remember that the rank theorem says that if $V$ and $W$ are finite dimensional vector spaces over $K$, and $\phi: V \to W$ is a linear map then:
	\begin{align*}
		\dim V &= \dim \ker(\phi) + \dim \phi(V)
	\end{align*}
	Let $\mathcal{A} = \{v_1, v_2, \dots, v_n\}$ be a basis for $ker(\phi)$ and let $\mathcal{B} = \{w_1, w_2, \dots, w_m\}$ be a basis for $\phi(V)$. Since $\mathcal{B} \subseteq \phi(V)$ there are $u_1, u_2, \dots, u_m$ such that $\phi(u_1) = w_1, \phi(u_2) = w_2, \dots, \phi(u_m) = w_m$. So, we can create the set:
	\begin{align*}
		\mathcal{C} &= \{v_1, v_2, \dots, v_n, u_1, u_2, \dots, u_m\}
	\end{align*}
	And we claim that this is a basis for $V$. For that, let's prove the two properties for that:
	\begin{itemize}
		\item Suppose that there are scalars $a_1, a_2, \dots, a_n, b_1, \dots, b_m$ such that:
		\begin{align*}
			a_1v_1 + a_2v_2 + \dots+  a_nv_n + b_1u_1 + b_2u_2 + \dots + b_mu_m &= 0\\
			a_1v_1 + a_2v_2 + \dots+ a_nv_n &= -b_1u_1 - b_2u_2 - \dots - b_mu_m\\
			\phi(a_1v_1) + \phi(a_2v_2) + \dots + \phi(a_nv_n) &= \phi(-b_1u_1) + \phi(-b_2u_2) + \dots + \phi(-b_mu_m)\\
			a_1\phi(v_1) + a_2\phi(v_2) + \dots + a_n\phi(v_n) &= -b_1\phi(u_1) -b_2\phi(u_2) - \dots - b_n\phi(u_m)\\
			a_10 + a_20 + \dots + a_n0 &= -b_1w_1 -b_2w_2 - \dots - b_mw_m\\
			0 &= -b_1w_1 - b_2w_2 - \dots - b_mw_m
		\end{align*}
		And since $\mathcal{B}$ is a basis then $b_1 = b_2 = \dots = b_m = 0$. And therefore we have that:
		\begin{align*}
			a_1v_1 + a_2v_2 + \dots + a_nv_n + b_1u_2 + b_2u_2 + \dots + b_mu_m &= 0\\
			a_1v_1 + a_2v_2 + \dots + a_nv_n &= 0
		\end{align*}
		And since $\mathcal{A}$ is a basis, then $a_1 = a_2 = \dots = a_n = 0$, and so $\mathcal{C}$ is linearly independent.

		\item Take $v \in V$, we want to prove it is a linear combination of elements of $\mathcal{C}$. So for that, we know that $\phi(v)$ is a linear combination of elements of $\mathcal{B}$:
		\begin{align*}
			b_1w_1 + b_2w_2 + \dots + b_mw_m &= \phi(v)\\
			b_1\phi(u_1) + b_2\phi(u_2) + \dots + b_m\phi(u_m) &= \phi(v)\\
			\phi(b_1u_1 + b_2u_2 + \dots + b_mu_m) &= \phi(v)\\
			\phi(b_1u_1 + b_2u_2 + \dots + b_mu_m) - \phi(v) &= 0\\
			\phi(b_1u_1 + b_2u_2 + \dots + b_mu_m - v) &= 0
		\end{align*}
		And since $b_1u_1 + b_2u_2 + \dots + b_mu_m - v \in \ker(\phi)$ we can derive a linear combination of the form:
		\begin{align*}
			a_1v_1 + a_2v_2 + \dots + a_nv_n &= b_1u_1 + b_2u_2 + \dots + b_mu_m - v\\
			a_1v_1 + a_2v_2 + \dots + a_nv_n - b_1u_1 - b_2u_2 - \dots - b_mu_m &= -v\\
			b_1u_1 + b_2u_2 + \dots + b_mu_m - a_1v_1 - a_2v_2 - \dots - a_nv_n &= v
		\end{align*}

		And so we have that $v$ is a linear combination of $\mathcal{C}$, so $Span(\mathcal{C}) = V$.
	\end{itemize}

	And that way we conclude that $\mathcal{C}$ is a basis for $V$ and note that $|\mathcal{C}| = |\mathcal{A}| + |\mathcal{B}|$, so $\dim V = \dim \ker(\phi) + \dim \phi(V)$.
}

\problem
Determine whether or not $\{(1, 1, 0), (2, 0, -1), (-3, 1, 1)\}$ is basis for $\mathbb{R}^3$

\solution{
	First, let's determine whenever it is linearly independent or not.
	\begin{itemize}
		\item Suppose that $a_1(1, 1, 0) + a_2(2, 0, -1) + a_3(-3, 1, 1) = 0$. So, if we add those vectors we would have:
		\begin{align*}
			a_1(1, 1, 0) + a_2(2, 0, -1) + a_3(-3, 1, 1) &= (a_1, a_1, 0) + (2a_2, 0, -a_2) + (-3a_3, a_3, a_3)\\
			&= (a_1 + 2a_2 - 3a_3, a_1 + a_3, -a_2 + a_3) = (0, 0, 0)
		\end{align*}
		So we would need that:
		\begin{align*}
			a_1 + 2a_2 - 3a_3 &= 0\\
			a_1 + a_3 &= 0\\
			a_3 - a_2 &= 0
		\end{align*}
		If we solve the last two equations for $a_1$ and $a_2$ we would have:
		\begin{align*}
			a_1 &= -a_3\\
			a_2 &= a_3
		\end{align*}
		And replacing in the first equation we would have:
		\begin{align*}
			a_1 + 2a_2 - 3a_3 &= 0\\
			-a_3 + 2a_3 - 3a_3 &= 0\\
			-2a_3 &= 0\\
			a_3 &= 0
		\end{align*}
		And so we conclude that $a_1 = a_2 = a_3 = 0$, so this set is linearly independent.

		\item Take now any vector $(x, y, z) \in \mathbb{R}^3$, we want to prove that we can always find a linear combination of the vectors that give us $(x, y, z)$. For that, suppose that there are such combinations, so:
		\begin{align*}
			a_1(1, 1, 0) + a_2(2, 0, -1) + a_3(-3, 1, 1) &= (x, y, z)\\
			(a_1, a_1, 0) + (2a_2, 0, -a_2) + (-3a_3, a_3, a_3) &= (x, y, z)\\
			(a_1 + 2a_2 - 3a_3, a_1 + a_3, a_3 - a_2) &= (x, y, z)
		\end{align*}
		And so we have:
		\begin{align*}
			a_1 + 2a_2 - 3a_3 &= x\\
			a_1 + a_3 &= y\\
			a_3 - a_2 &= z
		\end{align*}
		Then we have:
		\begin{align*}
			a_1 &= y-a_3\\
			a_2 &= a_3 - z
		\end{align*}
		And plugging into the first equation we have:
		\begin{align*}
			a_1 + 2a_2 - 3a_3 &= x\\
			y-a_3 + 2(a_3 - z) - 3a_3 &= x\\
			y-a_3 + 2a_3 - 2z - 3a_3 &= x\\
			y - 2z - 2a_3 &= x\\
			a_3 &= \frac{2z - x- y}{2}
		\end{align*}
		And plugging into the next equation:
		\begin{align*}
			a_1 &= y - a_3\\
			a_1 &= y - \frac{x+y-2z}{2}\\
			a_1 &= y + z - \frac{x}{2} + \frac{y}{2}\\
			a_1 &= \frac{3}{2} y + z - \frac{x}{2}
		\end{align*}

		And plugging into the last equation:
		\begin{align*}
			a_2 &= a_3 - z\\
			a_2 &= z - \frac{x}{2} - \frac{y}{2} - z\\
			a_2 &= \frac{-x-y}{2}
		\end{align*}
		And if you try this combination, you would get $(x, y, z)$ so we can see $Span(\{(1, 1, 0), (2, 0, -1), (-3, 1, 1)\}) = \mathbb{R}^3$.
	\end{itemize}

	And so we have proved that $\{(1, 1, 0), (2, 0, -1), (-3, 1, 1)\}$ is a basis for $\mathbb{R}^3$.
}

\problem
Let $\phi: V \to W$ be linear. Suppose that $v_1, \dots, v_n \in V$ are such that $\phi(v_1), \dots, \phi(v_n)$ are linearly independent in $W$. Show that $v_1, \dots, v_n$ are linearly independent.

\solution{
	For that, since $\phi(v_1), \dots, \phi(v_n)$ are linearly independent, we can assure that the equation:
	\begin{align*}
		a_1\phi(v_1) + a_2\phi(v_2) + \dots + a_n\phi(v_n) &= 0
	\end{align*}
	has only the trivial solution. Suppose that the equation:
	\begin{align*}
		b_1v_1 + b_2v_2 + \dots + b_nv_n &= 0
	\end{align*}
	has a solution that is not trivial. That this, we can assure that at least $b_1$ is not $0$. And if we apply to both sides the linear map $\phi$ we get:
	\begin{align*}
		\phi(b_1v_1 + b_2v_2 + \dots + b_nv_n) &= \phi(0)\\
		\phi(b_1v_1) + \phi(b_2v_2) + \dots + \phi(b_nv_n) &= 0\\
		b_1\phi(v_1) + b_2\phi(v_2) + \dots + b_n\phi(v_n) &= 0
	\end{align*}
	But this is a contradiction since this equation can only have the trivial solution. So we can conclude that $v_1, \dots v_n$.
}

\problem
If $\{v_1, \dots, v_n\}$ is a basis for $V$ and $\{w_1, \dots, w_m\}$ is a basis for $W$ then:
\begin{align*}
	\{(v_1, 0), \dots, (v_n, 0), (0, w_1), \dots, (0, w_n)\}
\end{align*}
is a basis for $V \oplus W$

\solution{
	We need to prove two things:
	\begin{itemize}
		\item First, to prove that this set is linearly independent, we need to show that the homogeneous equation has only the trivial solution. So we have:
		\begin{align*}
			a_1(v_1, 0) + a_2(v_2, 0) + \dots + a_n(v_n, 0) + b_1(0, w_1) + b_2(0, w_2) + \dots + b_n(0, w_n) &= (0, 0)\\
			(a_1v_1, 0) + (a_2v_2, 0) + \dots + (a_nv_n, 0) + (0, b_1w_1) + (0, b_2w_2) + \dots + (0, b_nw_n) &= (0, 0)\\
			(a_1v_1 + a_2v_2 + \dots  +a_nv_n, b_1w_1 + b_2w_2 + \dots + b_nw_n) &= (0, 0)
		\end{align*}
		And this means that:
		\begin{align*}
			a_1v_1 + a_2v_2 + \dots  +a_nv_n &= 0\\
			b_1w_1 + b_2w_2 + \dots + b_nw_n &= 0
		\end{align*}
		And since those vectors are basis for each vector space $a_1 = a_2 = \dots = a_n = b_1 = b_2 = \dots = b_n$.

		\item For an element $(v, w) \in V \oplus W$, we know that $v$ can be expressed as a linear combination $a_1v_1 + a_2v_2 + \dots + a_nv_n = v$, and also $w$ can be expressed as $b_1w_1 + b_2w_2 + \dots + b_nv_n = w$, so the combination of the vectors in our set will rise:
		\begin{align*}
			a_1(v_1, 0) + a_2(v_2, 0) + \dots + a_n(v_n, 0) + b_1(0, w_1) + b_2(0, w_2) + \dots + b_n(0, w_n) = (v, w)
		\end{align*}
	\end{itemize}
}

\problem
Let $W$ be a subspace of the finite-dimensional vector space $V$. Show that there is a subspace $U$ of $V$ such that $V \cong U \oplus W$.

\solution{
	For this, define $U$ as follows:
	\begin{align*}
		U &:= V \setminus W \cup \{0\}
	\end{align*}
	First, we need to prove that this is a subspace of $V$:
	\begin{quote}
		Note that for any $v \in U$ different from $0$ and any $c \in K$, if $cv \in W$ then $c^{-1}cv = v \in W$ which contradicts the definition of $U$. If $u, w \in U$ are not both $0$, and if $u + w \in W$ then that means that $u, w \in W$ since $W$ is closed over the operations, which again, contradicts the definition for $U$, so $u + w \in U$. 
	\end{quote}

	Now, we want to prove that this is an internal sum of $V$, so we have:
	\begin{itemize}
		\item If $w \in W$ and $u \in U$ are such that $w + u = 0$, then we would have $w = -u$, which means that $w \in U$ and also that $u = -w \in V$, which means that since its only common element is $0$, $u = w = 0$.
		\item For any element $v \in V$, there are two alternatives. If $v \in W$ then we can express $v$ as $v + 0$ and $0 \in U$. If $v \not\in W$ then $v \in U$ by definition and so $v = 0 + v$ with $0 \in W$. 
	\end{itemize}	

	And so we conclude that $U \oplus W$ is an internal sum of $V$.
}

\problem A linear map $\rho: V \to V$ is idempotent if $\rho\rho = \rho$. Show that $\rho$ acts as an identity over $\rho(V)$ if $\rho$ is idempotent.

\solution{
	For that, we want to prove that $\rho^2 = Id_{\rho(V)}$. For that, let $v \in \rho(V)$, we know that there is $w \in V$ such that $\rho(w) = v$. Now, if we apply again the function we would have:
	\begin{align*}
		\rho(\rho(w)) &= \rho(v)\\
		\rho(w) &= \rho(v)\\
		v &= \rho(v)
	\end{align*}
	So we conclude that $\rho^2 = Id_{\rho(V)}$.
}


\problem Decide if $\phi: \mathbb{R}^2 \to \mathbb{R}^2$ given by $\phi(x, y) = (x+y, 2x-y)$ is an isomorphism. If it is, find a formula for $\phi^{-1}(x, y)$ and prove they are inverses.

\solution{
	Suppose that for a vector $(a, b) \in \mathbb{R}^2$, exists $(x, y) \in \mathbb{R}^2$ whose image under $\phi$ is $(a, b)$. We would have:
	\begin{align*}
		\phi(x, y) &= (x+y, 2x-y) = (a, b)
	\end{align*}
	And so we can write the next equations:
	\begin{align*}
		x + y &= a\\
		2x-y &= b
	\end{align*}
	If we solve for $x$ in the first equation we would have:
	\begin{align*}
		x &= a - y
	\end{align*}
	And replacing in the second equation we would have:
	\begin{align*}
		2x-y = 2(a-y) - y &= b\\
		2a-2y-y &= b\\
		2a-3y &= b\\
		-3y &= b - 2a\\
		y &= \frac{2a-b}{3}
	\end{align*}
	And so if we plug in into the second equation we would have:
	\begin{align*}
		x &= a - y\\
		&= a - \frac{2a-b}{3}\\
		&= a - \frac{2a}{3} + \frac{b}{3}\\
		&= \frac{a}{3} + \frac{b}{3}\\
		&= \frac{a+b}{3}
	\end{align*}
	And so we would have:
	\begin{align*}
		\phi^{-1}(x, y) &= \left(\frac{x+y}{3}, \frac{2x-y}{3}\right)
	\end{align*}
	We can prove also that this indeed the inverse isomorphism by composing them:
	\begin{itemize}
		\item First, if we compose $\phi$ and $\phi^{-1}$ we would have:
		\begin{align*}
			\phi(\phi^{-1}(x, y)) &= \phi\left(\frac{x+y}{3}, \frac{2x-y}{3}\right)\\
			&= \left(\frac{x+y}{3} + \frac{2x-y}{3}, 2 \cdot \frac{x+y}{3} - \frac{2x-y}{3}\right)\\
			&= \left(\frac{3x}{3}, \frac{2x+2y}{3} + \frac{y-2x}{3}\right)\\
			&= \left(x, \frac{3y}{3}\right)\\
			&= (x, y)
		\end{align*}
		\item And now, if we compose $\phi^{-1}$ and $\phi$ we get:
		\begin{align*}
			\phi^{-1}(\phi(x, y)) &= \phi^{-1}\left(x+y, 2x-y\right)\\
			&= \left(\frac{x+y + 2x-y}{3}, \frac{2(x+y) - (2x-y)}{3}\right)\\
			&= \left(\frac{3x}{3}, \frac{2x+2y-2x+y}{3}\right)\\
			&= \left(x, \frac{3y}{3}\right)\\
			&= (x, y)
		\end{align*}
	\end{itemize}

	So we conclude that $\phi$ and $\phi^{-1}$ are inverses and so they are isomorphisms.
}

% TO DO
\problem Let $V$ be a vector space over a field $k$ and let $U, W$ be finite dimensional subspaces of $V$. Prove that both $U + W$ and $U \cap W$ are finite-dimensional subspaces of $V$ and that $$\dim(U + W) + \dim(U \cap W) = \dim U + \dim W$$

\solution{
	First, note that if $U \cap W$ is the empty set, then its dimension is $0$ and so it is finite-dimensional. Suppose it is not empty, so there is at least one $v \in U \cap W$. If we suppose that $\mathcal{B}$ is an infinite basis for $U \cap W$ then $v$ is a linear combination of the elements in $\mathcal{B}$. But also $v \in U$ but this would be a contradiction because this implies that $\mathcal{B}$ is linearly dependent and so it cannot be a basis for $U \cap W$.\\

	Now, we can find basis for each vector spaces as follows:
	\begin{align*}
		\mathcal{B} &= \{v_1, \dots, v_k\} (\text{Basis for $U \cap W$})\\
		\mathcal{B}_1 &= \{v_1, \dots, v_k, u_1, \dots, u_n\} (\text{Basis for $U$, since we can extend any basis})\\
		\mathcal{B}_2 &= \{v_1, \dots, v_k, w_1, \dots, w_m\} (\text{Basis for $W$,  since we can extend any basis})
	\end{align*}
	We are going to prove that $\mathcal{A} = \{v_1, \dots, v_k, u_1, \dots, u_n, w_1, \dots, w_m\}$ is a basis for $U + W$. 
	\begin{itemize}
		\item First, suppose that $a_1, \dots, a_k, b_1, \dots b_n, c_1, \dots, c_m$ are scalars in $k$ such that:
		\begin{align*}
			a_1v_1 + \dots + a_kv_k + b_1u_1 + \dots + b_nu_n + c_1w_1 + \dots + c_mw_m &= 0\\
		\end{align*}
		Suppose with no lose of generality that $a_1 \neq 0$, so we can express $v_1$ in the next way:
		\begin{align*}
			v_1 &= \frac{-a_2v_2 - \dots - a_kv_k - b_1u_1 - \dots - b_nu_n - c_1w_1 - \dots - c_mw_m}{a_1}
		\end{align*}
		But note that $v_1 \in U$ and $v_1 \in W$, so we can assure that

		\item For any element $v \in U + W$, we can express it as $u + w$ with $u \in U$ and $w \in W$. Now, for that we can express $u$ and $w$ as:
		\begin{align*}
			u &= a_1v_1 + a_2v_2 + \dots + a_kv_k + x_1u_1 + x_2u_2 + \dots + x_nu_n\\
			w &= b_1v_1 + b_2v_2 + \dots + b_kv_k + y_1w_1 + y_2w_2 + \dots + y_mw_m
		\end{align*}
		And if we add them up we get:
		\begin{align*}
			u + w &= a_1v_1 + a_2v_2 + \dots + a_kv_k + x_1u_1 + x_2u_2 + \dots + x_nu_n + b_1v_1 + b_2v_2 + \dots + b_kv_k + y_1w_1 + y_2w_2 + \dots + y_mw_m\\
			v &= (a_1+b_1)v_1 + (a_2 + b_2)v_2 + \dots + (a_k + b_k)v_k + x_1u_1 + x_2u_2 + \dots + x_nu_n  + y_1w_1 + y_2w_2 + \dots + y_mw_m
		\end{align*}
		And so we have proved that $Span(\mathcal{A}) = U + W$.
	\end{itemize}

	Therefore, we conclude that $\mathcal{A}$ is a basis for $U + W$. But since the basis for $U$ and the basis for $W$ includes both the basis for $U \cap W$, we need to extract it, so:
	\begin{align*}
		\dim (U + W) &= \dim U + \dim W - \dim(U \cap W)\\
		\dim (U + W) + \dim(U \cap W) &= \dim U + \dim W
	\end{align*}
}

\problem Let $\phi \in End(V)$ for a finite dimensional vector space $V$. Prove that $\phi$ is monic if and only if it is epic if and only if it is an isomorphism

\solution{
	Since $V$ is a finite dimensional vector space we can use the rank theorem to find the dimensions of the kernel, images and $V$.
	\begin{itemize}
		\item Suppose that $\phi$ is monic, so that $\ker (\phi) = \{0\}$. We would have then that $\dim \ker(\phi) = 0$ and so $\dim V = \dim \phi(V)$, and since $\phi(V) \subseteq V$ we conclude that $\phi(V) = V$ so that $\phi$ is epic.
		\item Suppose that $\phi$ is epic, so that $\phi(V) = V$. We would have then that $\dim\ker(\phi) = 0$, and since $\ker(\phi)$ is a subspace of $V$, if there would be a vector different from $0$ into the set, it would make a basis and so $\dim\ker(\phi) > 0$, so we would only have that $\ker(\phi) = \{0\}$ and so $\phi$ is monic.
		\item If we suppose that $\phi$ is monic or epic, we get the other one and so it is an isomorphism. If it is an isomorphism we are granted that it is monic and epic. 
	\end{itemize}
}

\problem If $\phi: \mathbb{R}^2 \to \mathbb{R}^2$ is defined as $\phi(x, y) = (x+y, 2x-y)$ then determine what is $p(\phi)$ when $p(x) = x^2 - 2x + 1$

\solution{
	Note that $\phi^2 = \phi \circ \phi$ and so $\phi^0 = Id_{\mathbb{R}^2}$, and we can write the polynomial as:
	\begin{align*}
		p(x) &= x^2 - 2x + 1x^0
	\end{align*}
	So if we apply it to $\phi$ we would have:
	\begin{align*}
		p(\phi(x, y)) &= \phi^2(x, y) - 2\phi(x, y) + 1\phi^0(x, y)\\
		&= \phi(x+y, 2x-y) - 2(x+y, 2x-y) + 1(x, y)\\
		&= (3x, 3y) + (-2x-2y, 2y-4x) + (x, y)\\
		&= (3x - 2x - 2y + x, 3y + 2y - 4x + y)\\
		&= (2x-2y, 6y-4x)
	\end{align*}
}


\end{document}