%========Top Matter=========== (fold)
\documentclass[final,expand]{problemset}
% For plot: pfdplots and settings
\usepackage{pgfplots}
\pgfplotsset{width=7cm,
	compat=newest,
	label style={font=\small},
	legend style={font=\small}
}
%========top_matter=========== (end)

\begin{document}
\heading[Caballero]{Sebastián Caballero}{2023}{Problem set}{30 Days of ML}

\section{Systems of linear equations, Matrices, and vector spaces}
\problem We consider $(\mathbb{R} \setminus \{-1\}, \star)$ where:
\begin{align*}
	a \star b &= a + ab + b
\end{align*}
for $a, b \in \mathbb{R} \setminus \{-1\}$. Show that this is an abelian group and solve $3 \star x \star x = 15$.

\solution{
	We prove this showing that the five axioms are true:
	\begin{itemize}
		\item First, suppose that $a + ab + b = -1$ and so we would have that $a(1 + b) + b = -1$, and so we have $a(1 + b) = - (1 + b)$ and since $b \neq 1$ we can assure that $1 + b \neq 0$ so $a = -1$ which is a contradiction to the hypothesis that $a \neq 1$, so $\star$ is closed under the set.
		\item Now, we want to prove associativity(We pay attention to the associativity in the real numbers)
		\begin{align*}
			a \star (b \star c) &= a \star (b + bc + c) & (a \star b) \star c &= (a + ab + b) \star c\\
			&= a + a(b + bc + c) + (b + bc + c) & &= (a + ab + b) + c(a + ab + b) + c\\
			&= a + ab + abc + ac + b + bc + c & &= a + ab + b + ac + abc + bc + c
		\end{align*}
		and so it is clear that they are the same.
		\item Note that the element $0$ is an identity element because:
		\begin{align*}
			a \star 0 &= a + a \cdot 0 + 0 & 0 \star a &= 0 + a \cdot 0 + a\\
			&= a & &= a
		\end{align*}
		\item The inverse element for $x$ is $\frac{-x}{1 + x}$ since:
		\begin{align*}
			x \star \frac{-x}{1 + x} &= x - x \cdot \frac{x}{1 + x} - \frac{x}{1 + x}\\
			&= x - \frac{x^2}{1 + x} - \frac{x}{1 + x}\\
			&= \frac{x + x^2}{1 + x} - \frac{x^2}{1 + x} - \frac{x}{1 + x}\\
			&= 0
		\end{align*}
		And the commutated case is the same, so it has inverses.
		\item The commutativity is a consequence of these properties in $\mathbb{R}$:
		\begin{align*}
			a \star b &= a + ab + b\\
			&= b + ba + a\\
			&= b \star a
		\end{align*}
	\end{itemize}

	And so we conclude that $(\mathbb{R} \setminus \{-1\}, \star)$ is an abelian group. For the equation, we do:
	\begin{align*}
		3 \star x \star x &= 15\\
		3 \star (2x + x^2) &= 15\\
		3 + 3(2x + x^2) + (2x + x^2) &= 15\\
		6x + 3x^2 + 2x + x^2 &= 12\\
		4x^2 + 8x &= 12\\
		x^2 + 2x &= 3\\
		x^2 + 2x - 3 &= 0\\
		(x -1)(x+3) &= 0 
	\end{align*}
	And therefore we conclude that $x = 1$ or $x = -3$. If we put this into the equation we have:
	\begin{align*}
		3 \star 1 \star 1 &= 3 \star 3 & 3 \star (-3 \star (-3)) &= 3 \star 3\\
		&= 3 + 9 + 3 & &= 3 + 9 + 3\\
		&= 15 & &= 15
	\end{align*}
	So the solutions are $x = 1$ and $x = -3$.
}

\problem Consider the set $\mathcal{G}$ of $3 \times 3$ matrices defined as follows:
\begin{align*}
	\mathcal{G} &= \left\{\begin{bmatrix}1 & x & z\\ 0 & 1 & y\\ 0 & 0 & 1 \end{bmatrix} \in \mathbb{R}^{3 \times 3} \, \, \vline \, \, x, y, z \in \mathbb{R}\right\}
\end{align*}
And we define $\cdot$ as the standard matrix multiplication. Is $(\mathcal{G}, \cdot)$ a group? If yes, is it abelian?

\solution{
	We prove this showing that the four axioms are true:
	\begin{itemize}
		\item First, it is closed under $\cdot$ since:
		\begin{align*}
			\begin{bmatrix}
				1 & x & z\\ 0 & 1 & y\\ 0 & 0 & 1
			\end{bmatrix} \cdot \begin{bmatrix}
				1 & a & c\\ 0 & 1 & b\\ 0 & 0 & 1
			\end{bmatrix} &= \begin{bmatrix}
				1 & a +x & c + bx + z\\ 0 & 1 & b+y\\ 0 & 0 & 1
			\end{bmatrix}
		\end{align*}
		And since the operations of addition and product of real numbers are closed, by definition the matrix is also in $\mathcal{G}$ so it is a closed operation.

		\item We can prove the associativity by taking three matrices and show that their product don't vary. 
		\begin{align*}
			\left(\begin{bmatrix}
				1 & x & z\\ 0 & 1 & y\\ 0 & 0 & 1
			\end{bmatrix} \cdot \begin{bmatrix}
				1 & a & c\\ 0 & 1 & b\\ 0 & 0 & 1
			\end{bmatrix}\right) \cdot \begin{bmatrix}
				1 & k & m\\ 0 & 1 & n\\ 0 & 0 & 1
			\end{bmatrix} &= \begin{bmatrix}
				1 & a +x & c + bx + z\\ 0 & 1 & b+y\\ 0 & 0 & 1
			\end{bmatrix} \cdot \begin{bmatrix}
				1 & a +x & c + bx + z\\ 0 & 1 & b+y\\ 0 & 0 & 1
			\end{bmatrix}\\
			&= \begin{bmatrix}
				1 & a + k + x& m + an +ax + c + bx + z\\
				0 & 1 & b + n + y\\
				0 & 0 & 1
			\end{bmatrix}
		\end{align*}
		And if we make the other option:
		\begin{align*}
			\begin{bmatrix}
				1 & x & z\\ 0 & 1 & y\\ 0 & 0 & 1
			\end{bmatrix} \cdot \left(\begin{bmatrix}
				1 & a & c\\ 0 & 1 & b\\ 0 & 0 & 1
			\end{bmatrix} \cdot \begin{bmatrix}
				1 & k & m\\ 0 & 1 & n\\ 0 & 0 & 1
			\end{bmatrix}\right) &= \begin{bmatrix}
				1 & x & z\\ 0 & 1 & y\\ 0 & 0 & 1
			\end{bmatrix} \cdot \begin{bmatrix}
				1 & a + k & m + an + c\\
				0 & 1 & n + b\\
				0 & 0 & 1
			\end{bmatrix}\\
			&= \begin{bmatrix}
				1 & a + k + x & m + an + c + xn+ bx\\
				0 & 1 & b + n + y\\
				0 & 0 & 1
			\end{bmatrix}
		\end{align*}
		And we can see that they are the same, so $\cdot$ is associative.

		\item Note that the matrix $I_3$ is also in the set since $0 \in \mathbb{R}$, and we know that any matrix $3 \times 3$ operated with $I_3$ is the same, so $I_3$ is the identity for $\mathcal{G}$.
		\item For finding the inverse matrix for an element in $\mathcal{G}$ we do the next:
		\begin{align*}
			\begin{bmatrix}
				1 & x & z &\vline & 1 & 0 & 0\\ 0 & 1 & y &\vline & 0 & 1 & 0\\ 0 & 0 & 1 &\vline & 0 & 0 &1
			\end{bmatrix} &\longrightarrow \begin{bmatrix}
				1 & x & 0 &\vline & 1 & 0 & -z\\ 0 & 1 & 0 &\vline & 0 & 1 & -y\\ 0 & 0 & 1 &\vline & 0 & 0 &1
			\end{bmatrix} \begin{matrix}
				\\ -zR_3 \\ -yR_3
			\end{matrix}\\
			&\longrightarrow \begin{bmatrix}
				1 & 0 & 0 &\vline & 1 & -x & xy-z\\ 0 & 1 & 0 &\vline & 0 & 1 & -y\\ 0 & 0 & 1 &\vline & 0 & 0 &1
			\end{bmatrix}\begin{matrix}
				-xR_2\\ \\ 
			\end{matrix}
		\end{align*}

		So the matrix done in the right side is the inverse of the matrix in $\mathcal{G}$. For that, note that:
		\begin{align*}
			\begin{bmatrix}
				1 & x & z\\ 0 & 1 & y\\ 0 & 0 & 1
			\end{bmatrix} \cdot \begin{bmatrix}
				1 & -x & xy-z\\ 0 & 1 & -y\\ 0 & 0 & 1
			\end{bmatrix} &= \begin{bmatrix}
				1 & 0 & 0\\ 0 & 1 & 0\\ 0 & 0 & 1
			\end{bmatrix}
		\end{align*}
		And the same is true for the converse, and since $-x, -y$ and $xy-z$ are also real numbers, we have shown the existence of inverses in $\mathcal{G}$.
	\end{itemize}

	Note that this group is not abelian since:
	\begin{align*}
		\begin{bmatrix}
			1 & x & z\\ 0 & 1 & y\\ 0 & 0 & 1
		\end{bmatrix} \cdot \begin{bmatrix}
			1 & a & c\\ 0 & 1 & b\\ 0 & 0 & 1
		\end{bmatrix} &= \begin{bmatrix}
			1 & a +x & c + bx + z\\ 0 & 1 & b+y\\ 0 & 0 & 1
		\end{bmatrix} & \begin{bmatrix}
			1 & a & c\\ 0 & 1 & b\\ 0 & 0 & 1
		\end{bmatrix} \cdot \begin{bmatrix}
			1 & x & z\\ 0 & 1 & y\\ 0 & 0 & 1
		\end{bmatrix} &= \begin{bmatrix}
			1 & a + x & z + ay + c\\ 0 & 1 & b + y\\ 0 & 0& 1
		\end{bmatrix}
	\end{align*}
	And if $bx \neq ay$ then they are not the same. So, the group is not abelian.
}

\problem If it is possible compute the next products.

\solution{
	The products are:
	\begin{enumerate}
		\item \begin{align*}
			\begin{bmatrix}
				1 & 2\\ 4 & 5\\ 7 & 8
			\end{bmatrix} \cdot \begin{bmatrix}
				1 & 1 & 0\\ 0 & 1 & 1\\ 1 & 0 & 1
			\end{bmatrix}
		\end{align*}

		This product is not possible since the first matrix is a $3 \times 2$ matrix and the other one is a $3 \times 3$ matrix, and hence $3 \neq 2$ we cannot operate it.

		\item \begin{align*}
			\begin{bmatrix}
				1 & 2 & 3\\ 4 & 5 & 6\\ 7 & 8 & 9
			\end{bmatrix} \cdot \begin{bmatrix}
				1 & 1 & 0\\  0 & 1 & 1\\ 1 & 0 & 1
			\end{bmatrix} &= \begin{bmatrix}
				4 & 3 & 5\\ 10 & 9 & 11\\ 16 & 15 & 17
			\end{bmatrix}
		\end{align*}

		\item \begin{align*}
			\begin{bmatrix}
				1 & 1 & 0\\  0 & 1 & 1\\ 1 & 0 & 1
			\end{bmatrix} \cdot \begin{bmatrix}
				1 & 2 & 3\\ 4 & 5 & 6\\ 7 & 8 & 9
			\end{bmatrix} &= \begin{bmatrix}
				5 & 7 & 9\\ 11 & 13 & 15\\ 8 & 10 & 12
			\end{bmatrix}
		\end{align*}

		\item \begin{align*}
			\begin{bmatrix}
				1 & 2 & 1 & 2\\ 4 & 1 & -1 & -4
			\end{bmatrix} \cdot \begin{bmatrix}
				 0 & 3\\ 1 & -1\\ 2 & 1 \\ 5 & 2
			\end{bmatrix} &= \begin{bmatrix}
				14 & 5\\ -21 & 2
			\end{bmatrix}
		\end{align*}

		\item \begin{align*}
			\begin{bmatrix}
				0 & 3\\ 1 & -1\\ 2 & 1 \\ 5 & 2
		   \end{bmatrix} \cdot \begin{bmatrix}
			1 & 2 & 1 & 2\\ 4 & 1 & -1 & -4
		\end{bmatrix} &= \begin{bmatrix}
			12 & 3 & -3 & -12\\
			-3 & 1 & 2 & 6\\
			6 & 5 & 1 & 0\\
			13 & 12 & 3 & 2
		\end{bmatrix}
		\end{align*}
	\end{enumerate}
}

\problem Find all the solutions in $x = \begin{bmatrix} x_1 \\ x_2\\ x_3\end{bmatrix} \in \mathbb{R}^3$ of the equation system $Ax = 12x$ where:
\begin{align*}
	A &= \begin{bmatrix}
		6 & 4 & 3\\ 6 & 0 & 9\\ 0 & 8 & 0
	\end{bmatrix}
\end{align*}
and $x_1 + x_2 + x_3 = 1$.

\solution{
	First, note that $Ax$ is:
	\begin{align*}
		\begin{bmatrix}
			6 & 4 & 3\\ 6 & 0 & 9\\ 0 & 8 & 0
		\end{bmatrix} \begin{bmatrix}
			x_1 \\ x_2 \\ x_3
		\end{bmatrix} &= \begin{bmatrix}
			6x_1 + 4x_2 + 3x_3\\ 6x_1 + 9x_3\\  8x_2
		\end{bmatrix}
	\end{align*}
	And so we want that:
	\begin{align*}
		\begin{bmatrix}
			6x_1 + 4x_2 + 3x_3\\ 6x_1 + 9x_3\\  8x_2
		\end{bmatrix} &= \begin{bmatrix}
			12x_1 \\ 12x_2 \\ 12x_3
		\end{bmatrix}
	\end{align*}
	So we have the equations\\
	\begin{align*}
		\begin{matrix}
			6x_1 &+& 4x_2 &+& 3x_3 &= 12x_1\\
			6x_1 &&&+& 9x_3 &= 12x_2\\
			&& 8x_2&& &= 12x_3
		\end{matrix}
	\end{align*}
	And so we end up with the system:
	\begin{align*}
		\begin{matrix}
			-6x_1 &+& 4x_2 &+& 3x_3 &= 0\\
			6x_1 &-& 12x_2 &+& 9x_3 &= 0\\
			&& 8x_2& -&12x_3 &= 0
		\end{matrix}
	\end{align*}
	Which can be expresed in the next matrix:
	\begin{align*}
		\begin{bmatrix}
			-6 & 4 & 3\\ 6 & -12 & 9\\ 0 & 8 & -12
		\end{bmatrix} &\longrightarrow \begin{bmatrix}
			-6 & 4 & 3\\ 0 & -8 & 12\\ 0 & 8 & -12
		\end{bmatrix}\begin{matrix}
			\\ -R_1\\ \\
		\end{matrix}\\
		&\longrightarrow \begin{bmatrix}
			-6 & 4 & 3\\ 0 & -8 & 12\\ 0 & 0 & 0
		\end{bmatrix} \begin{matrix}
			\\ \\ -R_2
		\end{matrix}\\
		&\longrightarrow \begin{bmatrix}
			-6 & 4 & 3\\ 0 & -2 & 3\\ 0 & 0 & 0
		\end{bmatrix} \begin{matrix}
			\\ \frac{1}{4}R_2 \\ \\
		\end{matrix}\\
		&\longrightarrow \begin{bmatrix}
			-6 & 6 & 0\\ 0 & -2 & 3\\ 0 & 0 & 0
		\end{bmatrix} \begin{matrix}
			-R_2 \\ \\ \\
		\end{matrix}
	\end{align*}

	Which lead us to the next equations:
	\begin{align*}
		-6x_1 + 6x_2 &= 0\\
		-2x_2 + 3x_3 &= 0
	\end{align*}
	From which we derive that:
	\begin{align*}
		x &= \begin{bmatrix}
			\frac{3}{2} x_3 \\ \frac{3}{2} x_3\\ x_3
		\end{bmatrix}
	\end{align*}

	And with the other condition, we must satisfy the equation as:
	\begin{align*}
		x_1 + x_2 + x_3 &= 1\\
		\frac{3}{2}x_3 + \frac{3}{2}x_3 + x_3 &= 1\\
		4x_3 &= 1\\
		x_3 &= \frac{1}{4}
	\end{align*}
}

\problem Which of the following sets are subsets of $\mathbb{R}^3$?
\solution{
	\begin{enumerate}
		\item $A = \{(\lambda, \lambda + \mu^3, \lambda-\mu^3) | \lambda, \mu \in \mathbb{R}\}$. Take two elements in $A$ and a scalar $c \in \mathbb{R}$, then:
		\begin{align*}
			c(\lambda_1, \lambda_1 + \mu_1^3, \lambda_1-\mu_1^3) + (\lambda_2, \lambda_2 + \mu_2^3, \lambda_2-\mu_2^3) &= (c\lambda_1, c\lambda_1 + c\mu_1^3, c\lambda_1 - c\mu_1^3) + (\lambda_2, \lambda_2 + \mu_2^3, \lambda_2-\mu_2^3)\\
			&= (c\lambda_1 + \lambda_2, (c\lambda_1 + \lambda_2) + (c\mu_1^3 + \mu_2^3), (c\lambda_1 + \lambda_2) - (c\mu_1^3 + \mu_2^3))
		\end{align*}
		And since $\sqrt[3]{c\mu_1^3 + \mu_2^3}$ will always be a real number, then the linear combination of these elements is in $A$ and so it is a subspace of $\mathbb{R}^3$

		\item $B = \{(\lambda^2, -\lambda^2, 0)|\lambda \in \mathbb{R}\}$. If you take $c = -1$ and two elements in $B$ we have:
		\begin{align*}
			-(\lambda_1^2, -\lambda_1^2, 0) + (\lambda_2^2, -\lambda_2^2, 0) &= (-\lambda_1^2, \lambda_1^2, 0) + (\lambda_2^2, -\lambda_2^2, 0)\\
			&= (\lambda_2^2 - \lambda_1^2, \lambda_1^2 - \lambda_2^2, 0)\\
		\end{align*}
		And if $\lambda_1^2 > \lambda_2^2$ then it is not defined its square and so it would not be an element of $B$. So $B$ is not a subspace of $\mathbb{R}^3$

		\item $C = \{(\lambda_1, \lambda_2, \lambda_3)| \lambda_1 - 2\lambda_2 + 3\lambda_3 = \gamma\}$ for a fixed $\gamma$. If we take two elements of $C$ we would have the equations:
		\begin{align*}
			\lambda_1 - 2\lambda_2 + 3\lambda_3 &= \gamma\\
			\psi_1- 2\psi_2 + 3\psi_3 &= \gamma
		\end{align*}
		And adding them up we get:
		\begin{align*}
			(\lambda_1 + \psi_1) - 2(\lambda_2 + \psi_2) + 3(\lambda_3 + \psi_3) &= 2\gamma
		\end{align*}
		So, unless $\gamma = 0$ we conclude that $C$ is not a subspace of $\mathbb{R}^3$.

		\item $D = \{(\lambda_1, \lambda_2, \lambda_3) | \lambda_2 \in \mathbb{Z}\}$. If we take an element $c \in \mathbb{R}$ such that $c$ is irrational and we multiply it by an element of $D$ then $c\lambda_2$ would not be an integer and so that element would not be on $D$. Therefore, $D$ is not a subspace of $\mathbb{R}^3$.
	\end{enumerate}
}

\problem Write the vector $\begin{bmatrix}1 \\ -2 \\ 5\end{bmatrix}$ as a linear combination of the vectors $x_1 = \begin{bmatrix}1 \\ 1 \\ 1\end{bmatrix}$, $x_2 = \begin{bmatrix}1 \\ 2 \\ 3\end{bmatrix}$ and $x_3 \begin{bmatrix}2\\-1\\1\end{bmatrix}$.

\solution{
	For that, let's write the matrix of the system of equations as:
	\begin{align*}
		\begin{bmatrix}
			1 & 1 & 2 & \vline & 1\\
			1 & 2 & -1 &\vline & -2\\
			1 & 3 & 1 &\vline & 5
		\end{bmatrix} &\longrightarrow \begin{bmatrix}
			1 & 1 & 2 & \vline & 1\\
			0 & 1 & -3 &\vline & -3\\
			0 & 2 & -1 &\vline & 4
		\end{bmatrix}\begin{matrix}
			\\ -R_1 \\ -R_1
		\end{matrix}\\
		&\longrightarrow \begin{bmatrix}
			1 & 1 & 2 & \vline & 1\\
			0 & 1 & -3 &\vline & -3\\
			0 & 0 & 5 &\vline & 10
		\end{bmatrix}\begin{matrix}
			\\  \\ -2R_2
		\end{matrix}\\
		&\longrightarrow \begin{bmatrix}
			1 & 1 & 2 & \vline & 1\\
			0 & 1 & -3 &\vline & -3\\
			0 & 0 & 1 &\vline & 2
		\end{bmatrix}\begin{matrix}
			\\  \\ \frac{1}{2}R_3
		\end{matrix}\\
		&\longrightarrow \begin{bmatrix}
			1 & 1 & 0 & \vline & -3\\
			0 & 1 & 0 &\vline & 3\\
			0 & 0 & 1 &\vline & 2
		\end{bmatrix}\begin{matrix}
			-2R_3\\ +3R_3 \\ \\
		\end{matrix}\\
		&\longrightarrow \begin{bmatrix}
			1 & 0 & 0 & \vline & -6\\
			0 & 1 & 0 &\vline & 3\\
			0 & 0 & 1 &\vline & 2
		\end{bmatrix}\begin{matrix}
			-R_2\\ \\ \\
		\end{matrix}\\
	\end{align*}
	And so we have the linear combination:
	\begin{align*}
		-6\begin{bmatrix}
			1\\1\\1
		\end{bmatrix} + 3\begin{bmatrix}
			1 \\2\\3
		\end{bmatrix} + 2\begin{bmatrix}
			2\\-1\\1
		\end{bmatrix} &= \begin{bmatrix}
			-6\\-6\\-6
		\end{bmatrix} + \begin{bmatrix}
			3 \\6\\9
		\end{bmatrix} + \begin{bmatrix}
			4\\-2\\2
		\end{bmatrix}\\
		&= \begin{bmatrix}
			1\\-2\\5
		\end{bmatrix}
	\end{align*}
}
\end{document}